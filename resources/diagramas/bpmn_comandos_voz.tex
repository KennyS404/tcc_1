% BPMN - Processo de Comandos de Voz Alexa
\begin{figura}{BPMN - Processo de execução de comandos de voz na integração Alexa}{O Autor}
\centering
\begin{tikzpicture}[scale=0.8]
% Definição de estilos
\tikzset{
  start/.style={circle, draw, fill=green!20, text width=1cm, text centered, minimum height=1cm},
  end/.style={circle, draw, fill=red!20, text width=1cm, text centered, minimum height=1cm},
  process/.style={rectangle, draw, fill=blue!20, text width=2.5cm, text centered, minimum height=0.8cm},
  decision/.style={diamond, draw, fill=yellow!20, text width=2cm, text centered, minimum height=0.8cm},
  gateway/.style={diamond, draw, fill=gray!20, text width=1.5cm, text centered, minimum height=0.8cm},
  note/.style={rectangle, draw, fill=white, text width=2.5cm, text centered, minimum height=0.6cm}
}

% Eventos de início e fim
\node[start] (start) at (0,8) {Início};
\node[end] (end) at (0,0) {Fim};

% Processos principais
\node[process] (voice_input) at (0,7) {Usuário fala\\comando para Alexa};
\node[process] (alexa_process) at (0,6) {Alexa processa\\comando de voz};
\node[process] (intent_recognition) at (0,5) {Reconhecimento de\\intenção (NLP)};
\node[decision] (valid_intent) at (0,4) {Intenção\\válida?};

% Processos de validação
\node[process] (auth_check) at (-3,3) {Verificar\\autenticação};
\node[process] (permission_check) at (3,3) {Verificar\\permissões};

% Gateway de paralelização
\node[gateway] (parallel_gateway) at (0,2.5) {+};

% Processos de execução
\node[process] (api_call) at (-3,1.5) {Chamar API\\Pollen};
\node[process] (data_processing) at (3,1.5) {Processar\\dados};

% Gateway de sincronização
\node[gateway] (sync_gateway) at (0,0.8) {+};

% Processos de resposta
\node[process] (format_response) at (0,0.4) {Formatar\\resposta SSML};

% Processos de erro
\node[process] (error_handling) at (-4,2) {Tratar erro\\e informar usuário};
\node[process] (help_response) at (4,2) {Fornecer ajuda\\sobre comandos};

% Fluxos principais
\draw[->] (start) -- (voice_input);
\draw[->] (voice_input) -- (alexa_process);
\draw[->] (alexa_process) -- (intent_recognition);
\draw[->] (intent_recognition) -- (valid_intent);

% Fluxos de decisão
\draw[->] (valid_intent) -- node[above] {Sim} (parallel_gateway);
\draw[->] (valid_intent) -- node[left] {Não} (help_response);

% Fluxos paralelos
\draw[->] (parallel_gateway) -- (auth_check);
\draw[->] (parallel_gateway) -- (permission_check);

% Fluxos de sincronização
\draw[->] (auth_check) -- (sync_gateway);
\draw[->] (permission_check) -- (sync_gateway);

% Fluxos de execução
\draw[->] (sync_gateway) -- (api_call);
\draw[->] (sync_gateway) -- (data_processing);

% Fluxo final
\draw[->] (api_call) -- (format_response);
\draw[->] (data_processing) -- (format_response);
\draw[->] (format_response) -- (end);

% Fluxos de erro
\draw[->] (help_response) -- (end);
\draw[->] (error_handling) -- (end);

% Anotações
\node[note] (note1) at (-5,6) {Comando: \textquotedblleft Alexa, pergunte\\ao Pollen sobre\\meus enxames\textquotedblright};
\node[note] (note2) at (5,4) {Exemplo de intenções:\\• ConsultarStatus\\• RegistrarAlimentacao\\• VerificarColheita};
\node[note] (note3) at (-5,1) {API REST:\\GET /enxame/user/\{id\}\\POST /alimentacao};
\node[note] (note4) at (5,1) {Processamento:\\• Validação de dados\\• Formatação de resposta\\• Geração de SSML};

\end{tikzpicture}
\label{fig:bpmn-comandos-voz}
\end{figura}