% Quadro - Processo de Comandos de Voz Alexa
\begin{quadro}{Processo de execução de comandos de voz na integração Alexa}{Elaborado pelo autor (2025)}
\label{quad:fluxo-comandos-voz}
\renewcommand{\arraystretch}{1.5}
\small
\begin{tabular}{|c|p{4cm}|p{8cm}|}
\hline
\textbf{Etapa} & \textbf{Processo} & \textbf{Descrição} \\ \hline

1 & Captura de Comando & O usuário fala um comando para o dispositivo Alexa. Exemplo: \textquotedblleft Alexa, pergunte ao Pollen quantos enxames eu tenho\textquotedblright \\ \hline

2 & Reconhecimento de Voz & A Alexa reconhece e processa o comando utilizando processamento de linguagem natural (NLP) \\ \hline

3 & Validação do Comando & O sistema verifica se o comando solicitado é válido e está disponível na Skill Pollen. Se inválido, a Alexa fornece ajuda sobre comandos disponíveis \\ \hline

4 & Verificação de Autenticação & O sistema verifica se o usuário está autenticado e vinculado a uma conta Pollen. Se não autenticado, solicita vinculação da conta \\ \hline

5 & Chamada à API & Com o usuário autenticado, o sistema realiza chamada à API do Pollen para buscar os dados solicitados (enxames, colheitas, dashboard, etc.) \\ \hline

6 & Processamento de Dados & Os dados retornados pela API são processados e formatados em linguagem SSML (Speech Synthesis Markup Language) para resposta em voz \\ \hline

7 & Resposta ao Usuário & A Alexa fornece a resposta em voz ao usuário com as informações solicitadas de forma clara e objetiva \\ \hline

\end{tabular}
\end{quadro}