% Quadros descritivos da estrutura de dados - Substituindo DER

\begin{quadro}{Entidades e atributos do banco de dados relevantes para integração Alexa}{Elaborado pelo autor (2025)}
\label{quad:entidades-atributos}
\renewcommand{\arraystretch}{1.4}
\small
\resizebox{\textwidth}{!}{
\begin{tabular}{|l|p{7.5cm}|p{5.5cm}|}
\hline
\textbf{Entidade} & \textbf{Atributos} & \textbf{Descrição} \\ \hline

\textbf{USER} & 
\underline{id}, email, password, planType, status
& 
Armazena informações dos usuários do aplicativo Pollen que utilizarão a integração com Alexa \\ \hline

\textbf{ENXAME} & 
\underline{id}, userId, especie, estadoOrigem, localizacao, identificador, forcaEnxame, createddate
& 
Representa as colmeias gerenciadas pelo apicultor, principal entidade consultada via comandos de voz \\ \hline

\textbf{ALIMENTACAO} & 
\underline{id}, enxameId, tipo, quantidade, observacao, createddate
& 
Registra alimentações fornecidas aos enxames, pode ser consultada e registrada via Alexa \\ \hline

\textbf{COLHEITA} & 
\underline{id}, enxameId, quantidade, tipo, observacao, createddate
& 
Armazena informações sobre colheitas realizadas, principal métrica consultada via comandos de voz \\ \hline

\textbf{MANEJO} & 
\underline{id}, enxameId, tipo, descricao, createddate
& 
Registra atividades de manejo realizadas nas colmeias (divisão, troca de caixa, etc.) \\ \hline

\textbf{REVISAO} & 
\underline{id}, enxameId, tipo, observacao, createddate
& 
Armazena revisões periódicas realizadas nos enxames para verificação de saúde \\ \hline

\textbf{DASHBOARD} & 
\underline{id}, userId, totalEnxames, totalColheita, ultimaAtividade
& 
Consolida dados gerais do apiário, fornecendo resumo geral via comando de voz \\ \hline

\end{tabular}
}
\end{quadro}

\vspace{1em}

\begin{quadro}{Relacionamentos entre as entidades do sistema}{Elaborado pelo autor (2025)}
\label{quad:relacionamentos}
\renewcommand{\arraystretch}{1.5}
\resizebox{\textwidth}{!}{
\begin{tabular}{|l|l|l|p{7cm}|}
\hline
\textbf{Entidade Origem} & \textbf{Relacionamento} & \textbf{Entidade Destino} & \textbf{Descrição} \\ \hline

USER & possui (1:N) & ENXAME & 
Um usuário pode possuir múltiplos enxames. A Alexa consulta os enxames do usuário autenticado. \\ \hline

USER & visualiza (1:1) & DASHBOARD & 
Cada usuário possui um dashboard único com estatísticas gerais do seu apiário. \\ \hline

ENXAME & tem (1:N) & ALIMENTACAO & 
Um enxame pode ter múltiplos registros de alimentação ao longo do tempo. \\ \hline

ENXAME & produz (1:N) & COLHEITA & 
Um enxame pode ter múltiplas colheitas registradas. Principal dado consultado via Alexa. \\ \hline

ENXAME & recebe (1:N) & MANEJO & 
Um enxame pode ter múltiplos manejos realizados (divisão, troca de caixa, etc.). \\ \hline

ENXAME & inspecionado (1:N) & REVISAO & 
Um enxame pode ter múltiplas revisões periódicas para verificação de saúde. \\ \hline

\end{tabular}
}
\end{quadro}

