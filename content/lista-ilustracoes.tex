% Lista de Ilustrações
\makeatletter
\ut@section[cap:lista-ilustracoes]{ut@capitulo}{\bfseries\MakeUppercase}{Lista de Ilustrações}
\makeatother

% Configuração de espaçamento conforme orientações do professor
\renewcommand{\baselinestretch}{1.5}\selectfont
\setlength{\parskip}{0pt}

\begin{flushleft}
\addcontentsline{toc}{section}{Lista de Ilustrações}

Ilustração 1 - Arquitetura do sistema Amazon Alexa Echo Dot \dotfill 45

Ilustração 2 - Modelo Cascata \dotfill 56

Ilustração 3 - Tela de criação de Skill Alexa \dotfill 83

Ilustração 4 - Tipos de linguagem aceitas pela Skill \dotfill 92

Ilustração 5 - Tela para testar Skill criada \dotfill 101

Ilustração 6 - Diagrama de Caso de Uso - Integração Alexa com Sistema Pollen \dotfill 86

Ilustração 7 - Diagrama de Entidade e Relacionamento (DER) - Entidades relevantes para integração Alexa \dotfill 89

Ilustração 8 - Modelo Conceitual do Banco de Dados - Tabelas relevantes para integração Alexa \dotfill 92

Ilustração 9 - Modelo Lógico do Banco de Dados - Estrutura física das tabelas \dotfill 95

Ilustração 10 - Diagrama de Classes - Arquitetura da integração Alexa com Pollen \dotfill 98

Ilustração 11 - Wireframes - Configuração da integração Alexa no aplicativo Pollen \dotfill 101

Ilustração 12 - Wireframes - Interface de comandos de voz da integração Alexa \dotfill 104

Ilustração 13 - BPMN - Processo de execução de comandos de voz na integração Alexa \dotfill 107

\end{flushleft}

% Restaurar configurações padrão
\renewcommand{\baselinestretch}{1.0}\selectfont
\setlength{\parskip}{\baselineskip}

