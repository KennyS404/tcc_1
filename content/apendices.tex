% APÊNDICES
\makeatletter
\ut@section[cap:apendices]{ut@capitulo}{\bfseries\MakeUppercase}{Apêndices}
\makeatother

% Configuração de espaçamento
\renewcommand{\baselinestretch}{1.5}\selectfont
\setlength{\parskip}{0pt}

\secao{Apêndice A - Questionário da Pesquisa com Usuários do Aplicativo Pollen}
\label{apendice:questionario}

Este apêndice apresenta o questionário aplicado aos usuários do aplicativo Pollen para validar a necessidade e interesse na integração com o assistente virtual Amazon Alexa. A pesquisa foi realizada através do Google Forms e enviada por e-mail para usuários ativos do aplicativo.

\vspace{1em}

\textbf{PESQUISA: POLLEN INTEGRAÇÃO COM ALEXA}

\vspace{0.5em}

\textbf{Objetivo:} Identificar o interesse dos usuários do aplicativo Pollen em uma integração com assistentes virtuais e mapear as funcionalidades mais desejadas.

\vspace{1em}

\textbf{Questão 1:} Você teria interesse em utilizar o aplicativo Pollen através de comandos de voz com assistentes virtuais como Alexa ou Google Assistant?

\begin{itemize}
    \item[($\,\,\,$)] Sim, tenho muito interesse
    \item[($\,\,\,$)] Sim, tenho interesse moderado
    \item[($\,\,\,$)] Talvez, dependendo das funcionalidades
    \item[($\,\,\,$)] Não tenho interesse
\end{itemize}

\vspace{1em}

\textbf{Questão 2:} Quais funcionalidades você gostaria de acessar por comandos de voz? (Múltipla escolha)

\begin{itemize}
    \item[($\,\,\,$)] Consultar idade da rainha de um enxame específico
    \item[($\,\,\,$)] Verificar força/estado do enxame
    \item[($\,\,\,$)] Consultar data da última divisão
    \item[($\,\,\,$)] Ver quantidade de enxames de determinada espécie
    \item[($\,\,\,$)] Consultar produção de mel de determinada espécie
    \item[($\,\,\,$)] Receber notificações e lembretes de manutenção
    \item[($\,\,\,$)] Registrar localização do meliponário
    \item[($\,\,\,$)] Obter passo a passo de cuidados com a colmeia
    \item[($\,\,\,$)] Consultar quantidade total de enxames
    \item[($\,\,\,$)] Registrar datas de divisões realizadas
    \item[($\,\,\,$)] Obter resumo geral do apiário
\end{itemize}

\vspace{1em}

\textbf{Questão 3:} Qual assistente virtual você prefere ou possui?

\begin{itemize}
    \item[($\,\,\,$)] Amazon Alexa
    \item[($\,\,\,$)] Google Assistant
    \item[($\,\,\,$)] Ambos
    \item[($\,\,\,$)] Não possuo nenhum
    \item[($\,\,\,$)] Outro: \_\_\_\_\_\_\_\_\_\_\_\_\_\_\_\_\_\_\_\_
\end{itemize}

\vspace{1em}

\textbf{Questão 4:} Em quais situações você utilizaria comandos de voz para acessar informações do Pollen? (Múltipla escolha)

\begin{itemize}
    \item[($\,\,\,$)] Durante visitas ao apiário/meliponário
    \item[($\,\,\,$)] Enquanto realizo manutenção nas colmeias
    \item[($\,\,\,$)] Quando estou com as mãos ocupadas
    \item[($\,\,\,$)] Para consultas rápidas no dia a dia
    \item[($\,\,\,$)] Antes de ir ao apiário para planejar atividades
    \item[($\,\,\,$)] Outro: \_\_\_\_\_\_\_\_\_\_\_\_\_\_\_\_\_\_\_\_
\end{itemize}

\vspace{1em}

\textbf{Questão 5:} Descreva outras funcionalidades ou comandos que você gostaria de ter disponíveis na integração com assistentes virtuais: (Resposta aberta)

\vspace{3em}
\noindent\rule{\textwidth}{0.4pt}

\vspace{3em}
\noindent\rule{\textwidth}{0.4pt}

\vspace{1em}

\textbf{Questão 6:} Qual a importância de poder acessar informações do Pollen sem usar as mãos?

\begin{itemize}
    \item[($\,\,\,$)] Muito importante
    \item[($\,\,\,$)] Importante
    \item[($\,\,\,$)] Moderadamente importante
    \item[($\,\,\,$)] Pouco importante
    \item[($\,\,\,$)] Não é importante
\end{itemize}

\vspace{1em}

\textbf{Resultados da Pesquisa:}

\begin{itemize}
    \item \textbf{Total de respondentes:} 5 usuários ativos do aplicativo Pollen
    \item \textbf{Taxa de interesse:} 100\% (5/5) manifestaram interesse na integração
    \item \textbf{Assistente preferido:} Amazon Alexa e Google Assistant
    \item \textbf{Funcionalidades mais solicitadas:}
    \begin{itemize}
        \item Consulta sobre idade da rainha
        \item Verificação da força do enxame
        \item Data da última divisão
        \item Quantidade de enxames por espécie
        \item Produção de mel por espécie
        \item Notificações e lembretes
    \end{itemize}
\end{itemize}

\newpage

\secao{Apêndice B - Exemplo de Estrutura de Código da Skill Alexa}
\label{apendice:codigo-skill}

Este apêndice apresenta um exemplo da estrutura de código planejada para a implementação da Skill Alexa integrada ao aplicativo Pollen, utilizando Node.js com TypeScript e seguindo as melhores práticas do Alexa Skills Kit.

\vspace{1em}

\begin{quadro}{JSON - Modelo de Interação da Skill Alexa}{Elaborado pelo autor}
\label{quad:codigo-interaction-model}
\begin{verbatim}
{
  "interactionModel": {
    "languageModel": {
      "invocationName": "pollen",
      "intents": [
        {
          "name": "ConsultaEnxamesIntent",
          "slots": [],
          "samples": [
            "quantos enxames eu tenho",
            "me mostre meus enxames",
            "consultar meus enxames",
            "quais são meus enxames",
            "informações sobre meus enxames"
          ]
        },
        {
          "name": "ConsultaIdadeRainhaIntent",
          "slots": [
            {
              "name": "enxame_id",
              "type": "AMAZON.NUMBER"
            }
          ],
          "samples": [
            "qual a idade da rainha do enxame {enxame_id}",
            "idade da rainha",
            "me fale sobre a rainha do enxame {enxame_id}",
            "quando nasceu a rainha"
          ]
        },
        {
          "name": "DashboardIntent",
          "slots": [],
          "samples": [
            "me mostre o resumo",
            "resumo do meu apiário",
            "dashboard",
            "como está meu apiário",
            "estatísticas gerais"
          ]
        }
      ]
    }
  }
}
\end{verbatim}
\end{quadro}

\newpage

\secao{Apêndice C - Roteiro de Testes Planejados}
\label{apendice:roteiro-testes}

Este apêndice apresenta o roteiro detalhado dos testes planejados para validar a integração Alexa+Pollen com usuários reais do aplicativo.

\vspace{1em}

\textbf{C.1 - Formulário de Avaliação de Funcionalidades}

O formulário a seguir será utilizado para avaliar se os requisitos funcionais foram atendidos:

\begin{quadro}{Formulário de Avaliação de Funcionalidades}{Elaborado pelo autor}
\label{quad:form-avaliacao-funcionalidades}
\renewcommand{\arraystretch}{1.5}
\resizebox{\textwidth}{!}{
\begin{tabular}{|l|p{8cm}|c|c|c|}
\hline
\textbf{RF} & \textbf{Descrição da Funcionalidade} & \textbf{Atendida} & \textbf{Parcial} & \textbf{Não Atendida} \\ \hline
RF001 & Autenticação de usuários via conta Pollen & ($\,\,\,$) & ($\,\,\,$) & ($\,\,\,$) \\ \hline
RF002 & Consulta de informações sobre enxames & ($\,\,\,$) & ($\,\,\,$) & ($\,\,\,$) \\ \hline
RF003 & Consulta de idade da rainha & ($\,\,\,$) & ($\,\,\,$) & ($\,\,\,$) \\ \hline
RF004 & Verificação da força do enxame & ($\,\,\,$) & ($\,\,\,$) & ($\,\,\,$) \\ \hline
RF005 & Consulta de data da última divisão & ($\,\,\,$) & ($\,\,\,$) & ($\,\,\,$) \\ \hline
RF006 & Consulta por espécie específica & ($\,\,\,$) & ($\,\,\,$) & ($\,\,\,$) \\ \hline
RF007 & Notificações e lembretes & ($\,\,\,$) & ($\,\,\,$) & ($\,\,\,$) \\ \hline
RF008 & Registro de localização & ($\,\,\,$) & ($\,\,\,$) & ($\,\,\,$) \\ \hline
RF009 & Orientações de cuidados (passo a passo) & ($\,\,\,$) & ($\,\,\,$) & ($\,\,\,$) \\ \hline
RF010 & Registro de divisões & ($\,\,\,$) & ($\,\,\,$) & ($\,\,\,$) \\ \hline
RF011 & Dashboard resumido & ($\,\,\,$) & ($\,\,\,$) & ($\,\,\,$) \\ \hline
RF012 & Comandos de ajuda & ($\,\,\,$) & ($\,\,\,$) & ($\,\,\,$) \\ \hline
RF013 & Configuração de preferências & ($\,\,\,$) & ($\,\,\,$) & ($\,\,\,$) \\ \hline
\end{tabular}
}
\end{quadro}

\vspace{1em}

\textbf{Critério de Aceitação:} A solução será considerada funcional se atingir média igual ou superior a 7,0 conforme a fórmula:

\begin{equation}
\text{MédiaFunc} = \frac{(\sum \text{QtdA} + \sum \text{QtdEP} \times 0,5)}{N \times \text{QtdAva}} \times 10
\end{equation}

\newpage

\textbf{C.2 - Formulário de Avaliação de Eficácia (Escala Likert)}

\begin{quadro}{Formulário de Avaliação de Eficácia}{Elaborado pelo autor}
\label{quad:form-avaliacao-eficacia}
\renewcommand{\arraystretch}{1.5}
\resizebox{\textwidth}{!}{
\begin{tabular}{|l|p{6cm}|c|c|c|c|c|}
\hline
\textbf{ID} & \textbf{Critério de Avaliação} & \textbf{CF} & \textbf{C} & \textbf{I} & \textbf{D} & \textbf{DF} \\ \hline
E1 & A integração Alexa+Pollen facilita o acesso a informações durante o trabalho no apiário & ($\,\,$) & ($\,\,$) & ($\,\,$) & ($\,\,$) & ($\,\,$) \\ \hline
E2 & Os comandos de voz são intuitivos e fáceis de usar & ($\,\,$) & ($\,\,$) & ($\,\,$) & ($\,\,$) & ($\,\,$) \\ \hline
E3 & A Alexa reconhece corretamente os comandos em português brasileiro & ($\,\,$) & ($\,\,$) & ($\,\,$) & ($\,\,$) & ($\,\,$) \\ \hline
E4 & As respostas fornecidas pela Alexa são claras e úteis & ($\,\,$) & ($\,\,$) & ($\,\,$) & ($\,\,$) & ($\,\,$) \\ \hline
E5 & O tempo de resposta da Skill é adequado & ($\,\,$) & ($\,\,$) & ($\,\,$) & ($\,\,$) & ($\,\,$) \\ \hline
E6 & A integração reduz a necessidade de usar o celular durante o manejo & ($\,\,$) & ($\,\,$) & ($\,\,$) & ($\,\,$) & ($\,\,$) \\ \hline
E7 & A autenticação e vinculação de conta foi simples & ($\,\,$) & ($\,\,$) & ($\,\,$) & ($\,\,$) & ($\,\,$) \\ \hline
E8 & Recomendaria a integração para outros apicultores & ($\,\,$) & ($\,\,$) & ($\,\,$) & ($\,\,$) & ($\,\,$) \\ \hline
E9 & A Skill atende às minhas necessidades de gestão apícola & ($\,\,$) & ($\,\,$) & ($\,\,$) & ($\,\,$) & ($\,\,$) \\ \hline

\end{tabular}
}
\end{quadro}

\vspace{1em}

\textbf{Legenda:}
\begin{itemize}
    \item \textbf{CF} - Concordo Fortemente (peso 5)
    \item \textbf{C} - Concordo (peso 4)
    \item \textbf{I} - Indeciso (peso 3)
    \item \textbf{D} - Discordo (peso 2)
    \item \textbf{DF} - Discordo Fortemente (peso 1)
\end{itemize}

\vspace{1em}

\textbf{Critério de Aceitação:} A solução será considerada eficaz se atingir média igual ou superior a 3,5 conforme a fórmula:

\begin{equation}
\text{MédiaEfi} = \frac{(\sum \text{CF} \times 5 + \sum \text{C} \times 4 + \sum \text{I} \times 3 + \sum \text{D} \times 2 + \sum \text{DF} \times 1)}{N \times \text{QtdAva}}
\end{equation}

% Restaurar configurações padrão
\renewcommand{\baselinestretch}{1.0}\selectfont
\setlength{\parskip}{\baselineskip}
