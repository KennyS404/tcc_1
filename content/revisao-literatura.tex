\capitulo{Revisão da literatura}
\label{cap:revisao-literatura}

\secao{Fundamentos Teóricos}

\subsecao{Assistentes Virtuais e Processamento de Linguagem Natural}

Assistentes virtuais são sistemas computacionais que usam Processamento de Linguagem Natural (NLP) e reconhecimento de voz para interpretar comandos e executar tarefas automatizadas. \textcite{rajanala2025} explicam que esses sistemas combinam técnicas de reconhecimento de fala, análise semântica e síntese de voz para criar interfaces conversacionais naturais.

O Processamento de Linguagem Natural é um campo da Inteligência Artificial que permite aos computadores compreender, interpretar e gerar linguagem humana. Em assistentes virtuais, o NLP transforma comandos de voz em ações executáveis, passando por etapas de reconhecimento automático de fala (ASR - Automatic Speech Recognition), compreensão de linguagem natural (NLU - Natural Language Understanding) e geração de linguagem natural (NLG - Natural Language Generation).

\subsecao{Amazon Alexa Skills Kit}

O Amazon Alexa Skills Kit (ASK) é um conjunto de ferramentas, APIs e documentação que permite desenvolvedores criarem habilidades personalizadas (skills) para a Alexa. Uma Skill é uma aplicação de voz que estende as capacidades nativas da Alexa, permitindo que ela execute tarefas específicas através de comandos de voz.

A arquitetura de uma Alexa Skill possui três componentes principais:

\begin{itemize}
    \item \textbf{Modelo de Interação}: Define como usuários interagem com a Skill através de intents (intenções), utterances (frases que invocam intents) e slots (parâmetros variáveis nos comandos).
    \item \textbf{Backend da Skill}: Geralmente implementado como função AWS Lambda ou endpoint HTTPS, processa as requisições da Alexa e retorna respostas apropriadas.
    \item \textbf{Account Linking}: Mecanismo de autenticação OAuth 2.0 que vincula contas de usuários de serviços terceiros à Skill.
\end{itemize}

O fluxo de processamento de um comando de voz ocorre em oito etapas: (1) captura do áudio pelo dispositivo Echo; (2) envio do áudio para servidores Amazon; (3) processamento de NLP e identificação do intent; (4) extração de slots; (5) invocação do backend da Skill; (6) processamento da lógica de negócio; (7) retorno da resposta em JSON com SSML (Speech Synthesis Markup Language); (8) síntese e reprodução da voz ao usuário.

\subsecao{APIs RESTful e Autenticação JWT}

APIs RESTful (Representational State Transfer) são interfaces de programação que seguem os princípios arquiteturais REST, usando métodos HTTP (GET, POST, PUT, DELETE) para realizar operações sobre recursos. Na integração Alexa+Pollen, a Skill acessará a API RESTful do Pollen para consultar e manipular dados apícolas.

JSON Web Token (JWT) é um padrão aberto (RFC 7519) para transmissão segura de informações entre partes. No sistema planejado, JWT será usado para autenticação e autorização, garantindo que apenas usuários autenticados acessem seus dados através da Alexa. O token JWT contém claims (declarações) sobre o usuário e é assinado digitalmente, permitindo verificação de integridade e autenticidade sem necessidade de armazenar sessões no servidor.

\secao{Soluções Similares}

O setor agrícola tem adotado assistentes virtuais e interfaces de voz para tornar operações em campo mais eficientes e acessíveis. Na apicultura, atividades como monitoramento da saúde das abelhas, produção de mel e análise de variáveis ambientais exigem acesso imediato a dados, frequentemente com as mãos ocupadas.

\textcite{rajanala2025} demonstram que um assistente por voz agrícola permite que agricultores façam consultas sobre saúde do solo, previsões meteorológicas e controle de pragas de modo intuitivo, facilitando interações em ambientes remotos ou quando as mãos estão ocupadas.

Embora não existam integrações específicas entre Alexa e aplicações de gestão apícola, há casos de sucesso em domínios relacionados que servem como referência.

\textcite{gento2019iot} desenvolveu um protótipo de agricultura inteligente baseado em tecnologias IOT. O sistema integra sensores para monitoramento de condições ambientais (temperatura, umidade, luz) e do solo, usando Raspberry Pi como unidade central de processamento. Similar à proposta deste trabalho, Gento Ribas implementou controle por voz através da Amazon Alexa, permitindo que usuários ativem dispositivos de irrigação e ventilação com comandos de voz. O protótipo usa o protocolo MQTT para comunicação com dispositivos wireless ESP32, demonstrando a viabilidade técnica de integrar assistentes virtuais com sistemas de gestão agrícola. \textcite{gento2019iot} afirma que "voice is the easiest instruction to use" \cite[p. 45]{gento2019iot}, justificando a escolha da Alexa para controle hands-free quando o uso de dispositivos móveis é limitado.

O RAImundo é outro exemplo relevante. Desenvolvido pela Embrapa em parceria com o Ministério da Agricultura e Pecuária (MAPA), o Ministério do Desenvolvimento Agrário e Agricultura Familiar (MDA) e a startup AZap.AI \cite{embrapa2024raimundo}, o assistente virtual usa inteligência artificial generativa. Lançado em 2024, oferece assistência técnica automatizada via WhatsApp para pequenos e médios agricultores, abordando temas como manejo agrícola, clima, mercado, pragas, solo e produção sustentável. A ferramenta usa linguagem natural e foi projetada para inclusão digital, adaptando as respostas à realidade local dos agricultores. Diferentemente das soluções comerciais, o RAImundo é gratuito e desenvolvido com validação técnica da Embrapa, demonstrando o potencial de assistentes virtuais para democratizar o acesso à informação técnica no setor agrícola brasileiro.

\subsecao{Gestão Apícola Digital}

A gestão apícola moderna envolve o monitoramento de múltiplos aspectos das colmeias:
\begin{itemize}
    \item \textbf{Monitoramento de saúde}: Verificação de sinais de doenças, parasitas e stress das abelhas.
    \item \textbf{Controle de produção}: Acompanhamento da produção de mel, pólen e outros produtos.
    \item \textbf{Gestão de enxames}: Controle de divisões, capturas e migrações de enxames.
    \item \textbf{Registro de colheitas}: Documentação de datas, quantidades e qualidade dos produtos.
    \item \textbf{Análise climática}: Correlação entre condições meteorológicas e comportamento das abelhas.
\end{itemize}

Aplicações móveis como o Pollen facilitam essas tarefas, permitindo que apicultores registrem e acessem informações sobre suas colmeias de forma organizada.

\secao{Diferenciais da Proposta}

Embora não existam integrações específicas entre assistentes virtuais e aplicações de gestão apícola, a análise de soluções similares evidencia a viabilidade desta proposta. A solução apresenta os seguintes diferenciais:

\begin{itemize}
    \item \textbf{Integração nativa}: Desenvolvimento específico para o Pollen, não uma solução genérica.
    \item \textbf{Comandos especializados}: Vocabulário adaptado para terminologia apícola brasileira.
    \item \textbf{Contexto otimizado}: Interface de voz projetada para uso em apiário com mãos ocupadas.
    \item \textbf{Funcionalidades específicas}: Comandos para consulta de status de enxames, registro de alimentação e verificação de colheitas.
    \item \textbf{Segurança e privacidade}: Autenticação robusta e controle de acesso adequado aos dados sensíveis.
\end{itemize}

O Quadro \ref{quad:comparacao-apps-apicolas} (apresentado na Introdução) mostra que nenhum aplicativo de gestão apícola atual oferece integração com assistentes virtuais consolidados como Amazon Alexa, representando uma oportunidade de inovação no setor.
















