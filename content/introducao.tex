\capitulo{Introdução}
\label{cap:introducao}

A Amazon Alexa está presente em mais de 100 milhões de dispositivos no mundo \cite{amazon2019alexa100m}. No Brasil, seu uso cresceu 50\% em 2023, com mais de 2 bilhões de comandos de voz executados pelos usuários brasileiros \cite{forbes2024}. Os dispositivos da linha Echo oferecem uma interface de voz natural para interação com diversos serviços. O ecossistema ``Alexa Skills'' permite que desenvolvedores criem funcionalidades personalizadas e as disponibilizem para usuários finais através de comandos de voz.

A apicultura moderna tem se tornado cada vez mais tecnológica. Aplicações móveis oferecem funcionalidades para controle de produção, manejo de enxames, registro de colheitas e monitoramento de saúde das abelhas. O aplicativo Pollen: Gestão de Colmeias é uma solução consolidada neste segmento, oferecendo uma plataforma completa para apicultores gerenciarem suas operações.

Este trabalho propõe uma integração entre a Alexa e o aplicativo Pollen. Com ela, apicultores poderão consultar informações sobre suas colmeias usando comandos de voz naturais. A proposta busca facilitar o acesso a dados importantes durante o trabalho no apiário e meliponário. Nesse ambiente, o uso de dispositivos móveis é limitado: o apicultor precisa usar equipamentos de proteção individual e manter as mãos livres para manipular as colmeias. Uma pesquisa preliminar com usuários do Pollen validou essa necessidade — 100\% dos respondentes manifestaram interesse na integração com assistentes virtuais.


\secao{Problema e justificativa}
\label{sec:problema-pesquisa-justificativa}

Assistentes virtuais têm se integrado cada vez mais com aplicações de gestão empresarial, oferecendo interfaces de voz naturais para acesso a informações importantes. Porém, essa tecnologia ainda é pouco explorada no setor apícola brasileiro, apesar da apicultura representar uma atividade econômica significativa no país.

O setor apícola brasileiro possui clima e flora propícios para o desenvolvimento, mas enfrenta limitações no uso de ferramentas tecnológicas. \textcite{silva2021} destacam que "o desenvolvimento tecnológico limitado do setor apícola, contando com pouca inovação na utilização de ferramentas e métodos produtivos, afeta diretamente a produção tanto em volume como em qualidade" \cite[p. 10]{silva2021}. Essa deficiência na gestão básica de sistemas produtivos impacta diretamente os níveis de produção.

O problema central desta pesquisa está no acesso a informações durante o trabalho no apiário. Nesse ambiente, o uso de dispositivos móveis é limitado: o apicultor usa equipamentos de proteção individual e precisa manter as mãos livres para manipular as colmeias. Atualmente, é necessário interromper as atividades para consultar informações no aplicativo móvel, impactando a eficiência do trabalho.

A integração com Alexa permitirá que apicultores acessem informações importantes usando comandos de voz simples. Será possível consultar produção de mel, status dos enxames, próximas atividades de manejo e estatísticas de produtividade sem interromper o fluxo de trabalho. A solução representa uma inovação no setor apícola, oferecendo uma interface mais natural e eficiente para gestão de colmeias.

\textcite{silva2021} desenvolveram um sistema baseado em machine learning para apoio à decisão no gerenciamento de produção apícola. Os autores demonstram que "torna-se importante o uso de mecanismos de ordenamento, gestão e tomada de decisão" \cite[p. 15]{silva2021} para melhor organizar as atividades apícolas. A pesquisa realizada com usuários do Pollen confirma essa necessidade — 100\% dos respondentes manifestaram interesse na integração, demonstrando sua relevância e potencial de adoção.

A integração com assistentes virtuais também contribui para a modernização da apicultura. A tecnologia se torna mais acessível e intuitiva para apicultores de diferentes níveis de familiaridade com dispositivos digitais, promovendo a adoção de ferramentas de gestão tecnológica no setor.



\secao{Objetivos}
\label{sec:objetivos}

\subsecao{Objetivo geral}
\label{ssec:objetivo-geral}

Planejar e desenvolver uma integração entre a Amazon Alexa e o aplicativo Pollen: Gestão de Colmeias, permitindo consultas por voz sobre dados apícolas e facilitando o acesso a informações durante o trabalho no apiário.

\subsecao{Objetivos específicos}
\label{ssec:objetivos-especificos}

\begin{itemize}
    \item Analisar a arquitetura e funcionalidades do aplicativo Pollen para identificar dados que podem ser consultados via voz.
    
    \item Desenvolver uma Alexa Skill personalizada usando o Amazon Alexa Skills Kit e Node.js.

    \item Implementar a comunicação entre a Alexa Skill e a API do Pollen com autenticação JWT.

    \item Definir intents e utterances em português brasileiro para consultas sobre colmeias, produção de mel e atividades de manejo.

    \item Implementar respostas em formato SSML para melhorar a experiência do usuário.

    \item Validar a integração com testes de usuários reais do Pollen, avaliando usabilidade e eficácia da solução.
\end{itemize}

O Quadro 1 compara aplicativos de gestão apícola que oferecem funcionalidades de voz ou integração com assistentes virtuais. A análise evidencia uma lacuna de mercado que a integração Alexa+Pollen busca preencher. As informações foram coletadas das páginas oficiais dos aplicativos e das lojas Google Play Store e Apple App Store em outubro de 2025.

\begin{quadro}{Comparação de aplicativos de gestão apícola com funcionalidades de voz}{Elaborado pelo autor com base em análise de mercado de aplicativos apícolas (2025)}
\label{quad:comparacao-apps-apicolas}
\renewcommand{\arraystretch}{1.5}
\resizebox{\textwidth}{!}{
\begin{tabular}{|l|c|c|c|c|c|}
    \hline
    \textbf{Aplicativo} & \textbf{Gestão de} & \textbf{Controle de voz} & \textbf{Integração} & \textbf{Idioma} & \textbf{Plataforma} \\
    \textbf{} & \textbf{Colmeias} & \textbf{hands-free} & \textbf{Alexa/Google} & \textbf{PT-BR} & \textbf{} \\ \hline
 
   
    Apiary Book* & Sim & Sim (Premium) & Não & Limitado & Multi-plataforma \\ \hline
   
    Abelheiro** & Sim & Sim (in-app) & Não & Sim & Mobile \\ \hline
    BeeHapp*** & Sim & Não & Não & Sim & Mobile \\ \hline
    \textbf{Pollen +} & \textbf{Sim} & \textbf{Sim (Alexa)} & \textbf{Sim} & \textbf{Sim} & \textbf{Multi-plataforma} \\
    \textbf{Alexa (Proposto)} & & & & & \\ \hline
\end{tabular}
}
\raggedright
\small
*Apiary Book disponível em: https://www.apiarybook.com (Acesso em: 01 out. 2025)\\
**Abelheiro disponível na Google Play Store (Acesso em: 01 out. 2025)\\
***BeeHapp disponível na Google Play Store (Acesso em: 01 out. 2025)
\end{quadro}

O Quadro \ref{quad:comparacao-apps-apicolas} mostra que, embora existam aplicativos de gestão apícola com funcionalidades de voz, nenhum oferece integração nativa com assistentes virtuais consolidados como Amazon Alexa ou Google Assistant. As soluções existentes usam sistemas próprios de reconhecimento de voz in-app, limitando a experiência do usuário e a adoção da tecnologia. A integração Alexa+Pollen representa uma inovação no setor, oferecendo aos apicultores brasileiros acesso a informações através de um ecossistema amplamente utilizado no mercado de assistentes virtuais.