\capitulo{Introdução}
\label{cap:introducao}

A Amazon Alexa é uma das assistentes virtuais mais utilizadas no mundo, estando presente em mais de 100 milhões de dispositivos. 
Os dispositivos da linha Echo são os mais comuns para uso pessoal, oferecendo uma interface de voz natural para interação com diversos serviços. 
No Brasil, o uso da Alexa cresceu 50\% em 2023, com mais de 2 bilhões de comandos de voz executados pelos usuários brasileiros, 
demonstrando a rápida adoção desta tecnologia no país \cite{forbes2024}. Uma das funcionalidades oferecidas pela Alexa é o ecossistema ``Alexa Skills'', 
que permite que desenvolvedores terceiros criem funcionalidades personalizadas e as disponibilizem para usuários finais através de comandos de voz.

No contexto da apicultura moderna, a gestão de colmeias tem se tornado cada vez mais tecnológica, com aplicações móveis oferecendo funcionalidades 
para controle de produção, manejo de enxames, registro de colheitas e monitoramento de saúde das abelhas. 
O aplicativo Pollen: Gestão de Colmeias representa uma solução consolidada neste segmento, oferecendo uma plataforma completa para apicultores 
gerenciarem suas operações de forma eficiente.

Assim sendo, este trabalho propõe o planejamento de uma integração entre o assistente virtual Alexa e o aplicativo Pollen, 
permitindo que apicultores consultem informações sobre suas colmeias através de comandos de voz naturais. 
A integração planejada visa facilitar o acesso a dados importantes durante o trabalho no apiário e meliponário, onde o uso de dispositivos móveis pode ser limitado, dado que o apicultor/meliponicultor precisa usar equipamentos de proteção individual e manter as mãos livres para manipulação das colmeias.
A necessidade desta integração foi validada através de pesquisa preliminar realizada com usuários do aplicativo Pollen, onde 100\% dos respondentes manifestaram interesse na integração com assistentes virtuais.


\secao{Delimitação do tema de pesquisa}
\label{sec:delimitacao-tema-pesquisa}

Este estudo propõe o planejamento de uma integração entre o assistente virtual Amazon Alexa e o aplicativo Pollen: Gestão de Colmeias, 
permitindo consultas por voz sobre dados apícolas. O projeto será planejado através da criação de uma Alexa Skill personalizada 
que se comunicará com a API do aplicativo Pollen para fornecer informações sobre colmeias, produção de mel, atividades de manejo e estatísticas e lembretes de manutenção da colmeia.

Para o planejamento da integração, será utilizada a plataforma Amazon Alexa Skills Kit, 
que oferece ferramentas e APIs para criação de skills personalizadas. A skill será planejada para ser desenvolvida utilizando Node.js 
e hospedada como uma função AWS Lambda, garantindo escalabilidade e confiabilidade no processamento das requisições.

A comunicação entre a Alexa Skill e o aplicativo Pollen será planejada para ser realizada através de chamadas HTTP para a API RESTful existente, 
utilizando autenticação JWT para garantir a segurança dos dados. A skill permitirá consultas sobre informações como 
número de colmeias, produção de mel, status dos enxames, próximas atividades de manejo e estatísticas de produtividade.

O planejamento seguirá as melhores práticas de desenvolvimento de Alexa Skills, incluindo o uso de intents bem definidos, 
utterances em português brasileiro e respostas em formato SSML para melhor experiência do usuário. 

\textbf{Intents} representam as intenções do usuário, ou seja, o que ele deseja fazer (ex: consultar enxames, registrar alimentação). 
\textbf{Utterances} são as diferentes formas como o usuário pode expressar uma intenção através de comandos de voz naturais 
(ex: "Alexa, pergunte ao Pollen sobre meus enxames" ou "Alexa, quantos enxames eu tenho?"). 
\textbf{SSML} (Speech Synthesis Markup Language) é uma linguagem de marcação que permite controlar a síntese de voz, 
melhorando a pronúncia e entonação das respostas da Alexa.

O projeto será planejado para ser testado com usuários reais do aplicativo Pollen para validar a usabilidade e eficácia da integração.




\secao{Problema de pesquisa e justificativa}
\label{sec:problema-pesquisa-justificativa}

A integração de assistentes virtuais com aplicações de gestão empresarial tem se tornado cada vez mais comum, 
oferecendo interfaces de voz naturais para acesso a informações importantes. No entanto, a implementação dessa tecnologia no setor apícola 
ainda é limitada, especialmente no contexto brasileiro, onde a apicultura representa uma atividade econômica significativa.

Segundo Silva et al. (2021), o setor apícola brasileiro, apesar de possuir boas características de clima e flora propícias para o desenvolvimento, 
sofre com limitações no uso de ferramentas tecnológicas, o que afeta diretamente os níveis de produção. Os autores destacam que 
"o desenvolvimento tecnológico limitado do setor apícola, contando com pouca inovação na utilização de ferramentas e métodos produtivos, 
afeta diretamente a produção tanto em volume como em qualidade", revelando uma deficiência significativa na gestão básica de sistemas produtivos.

O problema central desta pesquisa reside na necessidade de facilitar o acesso a informações sobre colmeias durante o trabalho no apiário, 
onde o uso de dispositivos móveis pode ser limitado devido às condições de trabalho, uso de equipamentos de proteção individual 
e necessidade de manter as mãos livres para manipulação das colmeias. Atualmente, apicultores precisam interromper suas atividades 
para consultar informações no aplicativo móvel, o que pode impactar a eficiência do trabalho.

A implementação de uma integração com Alexa poderá auxiliar apicultores a acessar informações importantes sobre suas colmeias 
através de comandos de voz simples, permitindo consultas sobre produção de mel, status dos enxames, próximas atividades de manejo 
e estatísticas de produtividade sem interromper o fluxo de trabalho. Esta solução representa uma inovação no setor apícola, 
oferecendo uma interface mais natural e eficiente para gestão de colmeias. 

A necessidade de ferramentas tecnológicas no setor apícola é corroborada por Silva et al. (2021), que desenvolveram um sistema baseado em 
machine learning para apoio à decisão no gerenciamento de produção apícola, demonstrando que "torna-se importante o uso de mecanismos de 
ordenamento, gestão e tomada de decisão" para uma melhor organização das atividades decorrentes da apicultura. A validação desta necessidade 
foi confirmada através de pesquisa realizada com usuários do aplicativo Pollen, onde 100\% dos respondentes manifestaram interesse na integração, 
demonstrando a relevância e potencial de adoção da solução proposta.

Além disso, a integração com assistentes virtuais pode contribuir para a modernização da apicultura, 
tornando a tecnologia mais acessível e intuitiva para apicultores de diferentes níveis de familiaridade com dispositivos digitais, 
promovendo a adoção de ferramentas de gestão tecnológica no setor.



\secao{Objetivos}
\label{sec:objetivos}
Nesta seção serão abordados os objetivos gerais e específicos a serem buscados no decorrer da execução do trabalho proposto.

\subsecao{Objetivo geral}
\label{ssec:objetivo-geral}

Propor o planejamento de uma integração entre o assistente virtual Amazon Alexa e o aplicativo Pollen: Gestão de Colmeias, permitindo consultas por voz sobre dados apícolas e facilitando o acesso a informações durante o trabalho no apiário.

\subsecao{Objetivos específicos}
\label{ssec:objetivos-especificos}

Os objetivos específicos do projeto seguem a ordem cronológica de execução das atividades, conforme descrito a seguir:

\begin{itemize}
    \item Analisar a arquitetura e funcionalidades do aplicativo Pollen para identificar dados que podem ser consultados via voz.
    
    \item Planejar o desenvolvimento de uma Alexa Skill personalizada utilizando o Amazon Alexa Skills Kit e Node.js.

    \item Propor o planejamento da comunicação entre a Alexa Skill e a API do aplicativo Pollen utilizando autenticação JWT.

    \item Definir intents e utterances em português brasileiro para consultas sobre colmeias, produção de mel e atividades de manejo.

    \item Planejar a implementação de respostas em formato SSML para melhor experiência do usuário com a assistente virtual.

    \item Propor metodologia de testes da integração com usuários reais do aplicativo Pollen para validar a usabilidade e eficácia da solução.
\end{itemize}

No Quadro 1 apresentam-se os verbos que auxiliam na formulação dos objetivos.

\begin{quadro}{Verbos que auxiliam na construção de objetivos}{Adaptado do Manual de Orientações Metodológicas do Centro Universitário de Jaraguá do Sul – UNERJ}
\label{quad:verbos-objetivos}
\renewcommand{\arraystretch}{1.3}
\resizebox{\textwidth}{!}{
\begin{tabular}{|c|c|c|c|c|c|}
    \hline
    \textbf{Conhecimento} & \textbf{Compreensão} & \textbf{Aplicação} & \textbf{Análise} & \textbf{Síntese} & \textbf{Avaliação} \\ \hline
    Definir & Compreender & Resolver & Identificar & Narrar & Sustentar \\
    Enunciar & Codificar & Interpretar & Distinguir & Expor & Justificar \\
    Citar & Deduzir & Expor & Descrever & Explicar & Criticar \\
    Relatar & Converter & Redigir & Diferenciar & Sumariar & Valorizar \\
    Referir & Descrever & Explicar & Relacionar & Esquematizar & Escolher \\
    Detalhar & Identificar & Usar & Separar & Compilar & Selecionar \\
    Expor & Definir & Manejar & Decompor & Construir & Verificar \\
    Identificar & Demonstrar & Aplicar & Examinar & Formular & Constatar \\
    Indicar & Distinguir & Empregar & Localizar & Compor & Comprovar \\
    Marcar & Interpretar & Utilizar & Abstrair & Projetar & Estimar \\
    Sublinhar & Explicar & Comprovar & Discriminar & Simplificar & Medir \\
    Enumerar & Expor & Demonstrar & Detalhar & Classificar & Revisar \\
    Listar & Exemplificar & Produzir & Detectar & Agrupar & Eleger \\
    Registrar & Concretizar & Praticar & Omitir & Distribuir & Decidir \\
    Especificar & Narrar & Relacionar & Dividir & Modificar & Concluir \\
    Mostrar & Argumentar & Dramatizar & Especificar & Reacomodar & Precisar \\
    Distinguir & Decodificar & Discriminar & Descobrir & Combinar & Provar \\
    Reconhecer & Relacionar & Representar & Reorganizar & Gerar & Comparar \\
    Definir & Extrapolar & Traçar & Estruturar & Opinar & Avaliar \\
    Organizar & Opinar & Localizar & Planejar & Demonstrar & Categorizar \\
    & Predizer & Operar & Conceber & Contrastar & Fundamentar \\
    & Generalizar & Ilustrar & Programar & Julgar & \\
    & Resumir & & & & \\
\hline
\end{tabular}
}
\end{quadro}