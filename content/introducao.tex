\capitulo{Introdução}
\label{cap:introducao}

A Amazon Alexa é uma das assistentes virtuais mais utilizadas no mundo, estando presente em mais de 100 milhões de dispositivos \cite{amazon2019alexa100m}.
Os dispositivos da linha Echo são os mais comuns para uso pessoal, oferecendo uma interface de voz natural para interação com diversos serviços. 
No Brasil, o uso da Alexa cresceu 50\% em 2023, com mais de 2 bilhões de comandos de voz executados pelos usuários brasileiros, 
demonstrando a rápida adoção desta tecnologia no país \cite{forbes2024}. Uma das funcionalidades oferecidas pela Alexa é o ecossistema ``Alexa Skills'', 
que permite que desenvolvedores terceiros criem funcionalidades personalizadas e as disponibilizem para usuários finais através de comandos de voz.

No contexto da apicultura moderna, a gestão de colmeias tem se tornado cada vez mais tecnológica, com aplicações móveis oferecendo funcionalidades 
para controle de produção, manejo de enxames, registro de colheitas e monitoramento de saúde das abelhas. 
O aplicativo Pollen: Gestão de Colmeias representa uma solução consolidada neste segmento, oferecendo uma plataforma completa para apicultores 
gerenciarem suas operações de forma eficiente.

Assim sendo, este trabalho propõe o planejamento de uma integração entre o assistente virtual Alexa e o aplicativo Pollen, 
permitindo que apicultores consultem informações sobre suas colmeias através de comandos de voz naturais. 
A integração planejada visa facilitar o acesso a dados importantes durante o trabalho no apiário e meliponário, onde o uso de dispositivos móveis pode ser limitado, dado que o apicultor/meliponicultor precisa usar equipamentos de proteção individual e manter as mãos livres para manipulação das colmeias.
A necessidade desta integração foi validada através de pesquisa preliminar realizada com usuários do aplicativo Pollen, onde 100\% dos respondentes manifestaram interesse na integração com assistentes virtuais.


\secao{Problema e justificativa}
\label{sec:problema-pesquisa-justificativa}

A integração de assistentes virtuais com aplicações de gestão empresarial tem se tornado cada vez mais comum, 
oferecendo interfaces de voz naturais para acesso a informações importantes. No entanto, a implementação dessa tecnologia no setor apícola 
ainda é limitada, especialmente no contexto brasileiro, onde a apicultura representa uma atividade econômica significativa.

Segundo \textcite{silva2021}, o setor apícola brasileiro, apesar de possuir boas características de clima e flora propícias para o desenvolvimento, 
sofre com limitações no uso de ferramentas tecnológicas, o que afeta diretamente os níveis de produção. Os autores destacam que 
"o desenvolvimento tecnológico limitado do setor apícola, contando com pouca inovação na utilização de ferramentas e métodos produtivos, 
afeta diretamente a produção tanto em volume como em qualidade" \cite[p. 10]{silva2021}, revelando uma deficiência significativa na gestão básica de sistemas produtivos.

O problema central desta pesquisa reside na necessidade de facilitar o acesso a informações sobre colmeias durante o trabalho no apiário, 
onde o uso de dispositivos móveis pode ser limitado devido às condições de trabalho, uso de equipamentos de proteção individual 
e necessidade de manter as mãos livres para manipulação das colmeias. Atualmente, apicultores precisam interromper suas atividades 
para consultar informações no aplicativo móvel, o que pode impactar a eficiência do trabalho.

A implementação de uma integração com Alexa poderá auxiliar apicultores a acessar informações importantes sobre suas colmeias 
através de comandos de voz simples, permitindo consultas sobre produção de mel, status dos enxames, próximas atividades de manejo 
e estatísticas de produtividade sem interromper o fluxo de trabalho. Esta solução representa uma inovação no setor apícola, 
oferecendo uma interface mais natural e eficiente para gestão de colmeias. 

A necessidade de ferramentas tecnológicas no setor apícola é corroborada por \textcite{silva2021}, que desenvolveram um sistema baseado em 
machine learning para apoio à decisão no gerenciamento de produção apícola, demonstrando que "torna-se importante o uso de mecanismos de 
ordenamento, gestão e tomada de decisão" \cite[p. 15]{silva2021} para uma melhor organização das atividades decorrentes da apicultura. A validação desta necessidade 
foi confirmada através de pesquisa realizada com usuários do aplicativo Pollen, onde 100\% dos respondentes manifestaram interesse na integração, 
demonstrando a relevância e potencial de adoção da solução proposta.

Além disso, a integração com assistentes virtuais pode contribuir para a modernização da apicultura, 
tornando a tecnologia mais acessível e intuitiva para apicultores de diferentes níveis de familiaridade com dispositivos digitais, 
promovendo a adoção de ferramentas de gestão tecnológica no setor.



\secao{Objetivos}
\label{sec:objetivos}
Nesta seção serão abordados os objetivos gerais e específicos a serem buscados no decorrer da execução do trabalho proposto.

\subsecao{Objetivo geral}
\label{ssec:objetivo-geral}

Planejar e desenvolver uma integração entre o assistente virtual Amazon Alexa e o aplicativo Pollen: Gestão de Colmeias, permitindo consultas por voz sobre dados apícolas e facilitando o acesso a informações durante o trabalho no apiário.

\subsecao{Objetivos específicos}
\label{ssec:objetivos-especificos}

Os objetivos específicos do projeto seguem a ordem cronológica de execução das atividades, conforme descrito a seguir:

\begin{itemize}
    \item Analisar a arquitetura e funcionalidades do aplicativo Pollen para identificar dados que podem ser consultados via voz.
    
    \item Desenvolver uma Alexa Skill personalizada utilizando o Amazon Alexa Skills Kit e Node.js.

    \item Implementar a comunicação entre a Alexa Skill e a API do aplicativo Pollen utilizando autenticação JWT.

    \item Definir intents e utterances em português brasileiro para consultas sobre colmeias, produção de mel e atividades de manejo.

    \item Implementar respostas em formato SSML para melhor experiência do usuário com a assistente virtual.

    \item Validar a integração através de testes com usuários reais do aplicativo Pollen para avaliar a usabilidade e eficácia da solução.
\end{itemize}

No Quadro 1 apresenta-se uma comparação entre aplicativos de gestão apícola que oferecem funcionalidades de comandos por voz ou integração com assistentes virtuais, evidenciando a lacuna de mercado que a integração Alexa+Pollen visa preencher. As informações foram coletadas através de análise das páginas oficiais dos aplicativos e lojas de aplicativos (Google Play Store e Apple App Store) em outubro de 2025.

\begin{quadro}{Comparação de aplicativos de gestão apícola com funcionalidades de voz}{Elaborado pelo autor com base em análise de mercado de aplicativos apícolas (2025)}
\label{quad:comparacao-apps-apicolas}
\renewcommand{\arraystretch}{1.5}
\resizebox{\textwidth}{!}{
\begin{tabular}{|l|c|c|c|c|c|}
    \hline
    \textbf{Aplicativo} & \textbf{Gestão de} & \textbf{Controle de voz} & \textbf{Integração} & \textbf{Idioma} & \textbf{Plataforma} \\
    \textbf{} & \textbf{Colmeias} & \textbf{hands-free} & \textbf{Alexa/Google} & \textbf{PT-BR} & \textbf{} \\ \hline
 
   
    Apiary Book* & Sim & Sim (Premium) & Não & Limitado & Multi-plataforma \\ \hline
   
    Abelheiro** & Sim & Sim (in-app) & Não & Sim & Mobile \\ \hline
    BeeHapp*** & Sim & Não & Não & Sim & Mobile \\ \hline
    \textbf{Pollen +} & \textbf{Sim} & \textbf{Sim (Alexa)} & \textbf{Sim} & \textbf{Sim} & \textbf{Multi-plataforma} \\
    \textbf{Alexa (Proposto)} & & & & & \\ \hline
\end{tabular}
}
\raggedright
\small
*Apiary Book disponível em: https://www.apiarybook.com (Acesso em: 01 out. 2025)\\
**Abelheiro disponível na Google Play Store (Acesso em: 01 out. 2025)\\
***BeeHapp disponível na Google Play Store (Acesso em: 01 out. 2025)
\end{quadro}

Conforme demonstrado no Quadro \ref{quad:comparacao-apps-apicolas}, embora existam aplicativos de gestão apícola com funcionalidades de voz, nenhum deles oferece integração nativa com assistentes virtuais consolidados como Amazon Alexa ou Google Assistant. As soluções existentes utilizam sistemas próprios de reconhecimento de voz in-app, o que limita a experiência do usuário e a adoção da tecnologia. A proposta de integração Alexa+Pollen representa uma inovação no setor, oferecendo aos apicultores brasileiros acesso a informações através de um ecossistema amplamente utilizado e consolidado no mercado de assistentes virtuais.