\capitulo{Cronograma de Execução}
\label{cap:cronograma}

Este capítulo apresenta o cronograma detalhado de planejamento do projeto de integração entre Amazon Alexa e o aplicativo Pollen, organizando as atividades em fases sequenciais conforme a metodologia em cascata adotada.

\secao{Atividades e Marcos do Projeto}
\label{sec:atividades-marcos-projeto}

O cronograma do projeto está organizado em 6 meses, divididos em fases que correspondem às etapas da metodologia em cascata para planejamento. Cada fase possui marcos específicos e entregas definidas.

\subsecao{Fase 1 - Análise e Especificação de Requisitos (Mês 1)}

\textbf{Objetivo:} Definir e especificar todos os requisitos funcionais e não funcionais do sistema.

\textbf{Atividades:}
\begin{itemize}
    \item Análise do aplicativo Pollen e identificação de funcionalidades
    \item Entrevistas com apicultores usuários do aplicativo
    \item Especificação de requisitos funcionais
    \item Especificação de requisitos não funcionais
    \item Validação dos requisitos com stakeholders
\end{itemize}

\textbf{Entregas:}
\begin{itemize}
    \item Documento de requisitos funcionais
    \item Documento de requisitos não funcionais
    \item Relatório de análise do aplicativo Pollen
\end{itemize}

\subsecao{Fase 2 - Projeto e Arquitetura (Mês 2)}

\textbf{Objetivo:} Desenvolver a arquitetura do sistema e os diagramas de modelagem.

\textbf{Atividades:}
\begin{itemize}
    \item Definição da arquitetura do sistema
    \item Criação do diagrama de caso de uso
    \item Desenvolvimento do diagrama de classes
    \item Criação do diagrama de sequência
    \item Desenvolvimento do modelo de dados
    \item Criação de wireframes das interfaces
\end{itemize}

\textbf{Entregas:}
\begin{itemize}
    \item Documento de arquitetura do sistema
    \item Diagramas de modelagem UML
    \item Modelo de dados
    \item Wireframes das interfaces
\end{itemize}

\subsecao{Fase 3 - Planejamento da Implementação (Mês 3)}

\textbf{Objetivo:} Planejar a implementação técnica da solução.

\textbf{Atividades:}
\begin{itemize}
    \item Definição da stack tecnológica
    \item Planejamento da estrutura do código
    \item Definição dos padrões de desenvolvimento
    \item Planejamento da integração com APIs
    \item Definição da estratégia de testes
\end{itemize}

\textbf{Entregas:}
\begin{itemize}
    \item Documento de especificação técnica
    \item Plano de implementação
    \item Estratégia de testes
\end{itemize}

\subsecao{Fase 4 - Planejamento da Implementação (Meses 4-5)}

\textbf{Objetivo:} Planejar a implementação da solução conforme especificado.

\textbf{Atividades:}
\begin{itemize}
    \item Planejamento do desenvolvimento da Alexa Skill
    \item Planejamento da implementação da integração com API Pollen
    \item Planejamento do desenvolvimento das funcionalidades de voz
    \item Planejamento da implementação da autenticação JWT
    \item Planejamento do desenvolvimento das respostas SSML
    \item Planejamento dos testes unitários
    \item Planejamento dos testes de integração
\end{itemize}

\textbf{Entregas:}
\begin{itemize}
    \item Documentação de planejamento da Alexa Skill
    \item Documentação técnica detalhada
    \item Plano de testes
\end{itemize}

\subsecao{Fase 5 - Planejamento de Testes e Validação (Mês 6)}

\textbf{Objetivo:} Planejar a validação da solução com usuários reais.

\textbf{Atividades:}
\begin{itemize}
    \item Planejamento dos testes com usuários apicultores
    \item Planejamento da coleta de feedback
    \item Planejamento da análise de usabilidade
    \item Planejamento da correção de problemas identificados
    \item Finalização da documentação
\end{itemize}

\textbf{Entregas:}
\begin{itemize}
    \item Plano de testes com usuários
    \item Plano de análise de usabilidade
    \item Documentação final do projeto
\end{itemize}

\secao{Cronograma Detalhado}
\label{sec:cronograma-detalhado}

O Quadro \ref{quad:cronograma-detalhado} apresenta o cronograma detalhado com as atividades, responsáveis e prazos.

\begin{quadro}{Cronograma Detalhado de Execução do Projeto}{Elaborado pelo autor}
\label{quad:cronograma-detalhado}
\renewcommand{\arraystretch}{1.3}
\resizebox{\textwidth}{!}{
\begin{tabular}{|c|c|c|c|c|c|c|}
    \hline
    \textbf{Atividade} & \textbf{Mês 1} & \textbf{Mês 2} & \textbf{Mês 3} & \textbf{Mês 4} & \textbf{Mês 5} & \textbf{Mês 6} \\ \hline
    Análise do Pollen & X & & & & & \\ \hline
    Especificação de Requisitos & X & & & & & \\ \hline
    Definição da Arquitetura & & X & & & & \\ \hline
    Diagramas UML & & X & & & & \\ \hline
    Wireframes & & X & & & & \\ \hline
    Especificação Técnica & & & X & & & \\ \hline
    Planejamento da Skill & & & & X & X & \\ \hline
    Planejamento da Integração & & & & X & X & \\ \hline
    Planejamento dos Testes & & & & & X & \\ \hline
    Planejamento dos Testes com Usuários & & & & & & X \\ \hline
    Documentação Final & & & & & & X \\ \hline
\end{tabular}
}
\end{quadro}

\secao{Recursos Necessários}
\label{sec:recursos-necessarios}

Para a execução do projeto, serão necessários os seguintes recursos:

\subsecao{Recursos Humanos}
\begin{itemize}
    \item Desenvolvedor principal (1 pessoa)
    \item Apicultores para testes (5-10 pessoas)
    \item Orientador do projeto (1 pessoa)
\end{itemize}

\subsecao{Recursos Técnicos}
\begin{itemize}
    \item Computador para desenvolvimento
    \item Dispositivo Amazon Echo para testes
    \item Acesso à AWS (Amazon Web Services)
    \item Acesso à API do aplicativo Pollen
    \item Ferramentas de desenvolvimento (IDE, versionamento)
\end{itemize}

\subsecao{Recursos Financeiros}
\begin{itemize}
    \item Custos com AWS Lambda (estimativa: R\$ 50,00/mês)
    \item Custos com testes e validação (estimativa: R\$ 200,00)
    \item Total estimado: R\$ 500,00
\end{itemize}