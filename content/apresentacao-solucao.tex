\capitulo{Apresentação da Solução}
\label{cap:apresentacao-solucao}

\secao{Requisitos do Sistema}
\label{sec:requisitos-sistema}

\subsecao{Requisitos Funcionais}

Os requisitos funcionais especificam as funcionalidades que o sistema deve oferecer, baseados na pesquisa preliminar com usuários do Pollen. O Quadro \ref{quad:requisitos-funcionais} apresenta os 13 requisitos funcionais da integração Alexa+Pollen.

\begin{quadro}{Requisitos Funcionais da integração Alexa+Pollen}{Elaborado pelo autor (2025)}
\label{quad:requisitos-funcionais}
\renewcommand{\arraystretch}{1.4}
\small
\resizebox{\textwidth}{!}{
\begin{tabular}{|l|p{3.5cm}|p{9cm}|}
\hline
\textbf{ID} & \textbf{Nome do Requisito} & \textbf{Descrição} \\ \hline

RF001 & Autenticação & O sistema deve permitir que usuários autentiquem-se na Skill utilizando suas credenciais do aplicativo Pollen \\ \hline

RF002 & Consulta de Enxames & O sistema deve permitir consultar informações sobre os enxames do usuário através de comandos de voz \\ \hline

RF003 & Consulta de Idade da Rainha & O sistema deve permitir consultar a idade da rainha de enxames específicos \\ \hline

RF004 & Consulta de Força do Enxame & O sistema deve permitir verificar a força/estado do enxame \\ \hline

RF005 & Consulta de Data da Última Divisão & O sistema deve permitir consultar quando foi realizada a última divisão do enxame \\ \hline

RF006 & Consulta por Espécie & O sistema deve permitir consultar quantidade de enxames e produção de mel por espécie específica \\ \hline

RF007 & Notificações e Lembretes & O sistema deve fornecer lembretes de manutenção da colmeia e alimentação \\ \hline

RF008 & Registro de Localização & O sistema deve permitir registrar a localização do Meliponário \\ \hline

RF009 & Passo a Passo de Cuidados & O sistema deve fornecer orientações passo a passo sobre cuidados com a colmeia \\ \hline

RF010 & Registro de Divisões & O sistema deve permitir registrar datas de divisões realizadas \\ \hline

RF011 & Dashboard Resumido & O sistema deve fornecer um resumo geral do apiário do usuário \\ \hline

RF012 & Comandos de Ajuda & O sistema deve fornecer ajuda sobre comandos disponíveis \\ \hline

RF013 & Configuração de Usuário & O sistema deve permitir configurar preferências do usuário \\ \hline

\end{tabular}
}
\end{quadro}

\subsecao{Requisitos Não Funcionais}

Os requisitos não funcionais definem as restrições e qualidades que o sistema deve possuir. O Quadro \ref{quad:requisitos-nao-funcionais} apresenta os 8 requisitos não funcionais, organizados por categoria com critérios de aceitação definidos.

\begin{quadro}{Requisitos Não Funcionais da integração Alexa+Pollen}{Elaborado pelo autor (2025)}
\label{quad:requisitos-nao-funcionais}
\renewcommand{\arraystretch}{1.4}
\small
\resizebox{\textwidth}{!}{
\begin{tabular}{|l|p{2.8cm}|p{5.5cm}|p{4cm}|}
\hline
\textbf{ID} & \textbf{Categoria} & \textbf{Descrição} & \textbf{Critério de Aceitação} \\ \hline

RNF001 & Performance & O sistema deve responder a comandos de voz rapidamente & Tempo de resposta máximo de 3 segundos \\ \hline

RNF002 & Disponibilidade & O sistema deve estar disponível continuamente & Uptime mínimo de 99\% \\ \hline

RNF003 & Segurança & O sistema deve garantir proteção dos dados & Utilizar autenticação JWT e comunicação HTTPS \\ \hline

RNF004 & Usabilidade & O sistema deve reconhecer comandos em português brasileiro & Precisão mínima de 90\% no reconhecimento \\ \hline

RNF005 & Escalabilidade & O sistema deve suportar múltiplos usuários simultâneos & Suportar até 1000 usuários simultâneos \\ \hline

RNF006 & Compatibilidade & O sistema deve funcionar em dispositivos Amazon Echo & Compatível com Echo 2ª geração ou superior \\ \hline

RNF007 & Manutenibilidade & O código deve seguir padrões de desenvolvimento & Código documentado e seguindo padrões definidos \\ \hline

RNF008 & Confiabilidade & O sistema deve tratar erros adequadamente & Feedback apropriado em todas as situações de erro \\ \hline

\end{tabular}
}
\end{quadro}

\secao{Modelagem do Sistema}
\label{sec:modelagem-sistema}

\subsecao{Diagrama de Caso de Uso}

O Diagrama de Caso de Uso apresenta as interações entre os atores (Apicultor e Amazon Alexa) e o sistema, definindo as funcionalidades principais da integração.

% Diagrama de Caso de Uso - Integração Alexa com Pollen
\begin{figura}{Diagrama de Caso de Uso - Integração Alexa com Sistema Pollen}{O Autor}
\centering
\begin{tikzpicture}[scale=0.8]
% Definição de estilos
\tikzset{
  actor/.style={rectangle, draw, fill=blue!20, text width=2cm, text centered, minimum height=1cm},
  usecase/.style={ellipse, draw, fill=yellow!20, text width=2.5cm, text centered, minimum height=0.8cm},
  system/.style={rectangle, draw, fill=green!20, text width=8cm, text centered, minimum height=6cm}
}

% Sistema Pollen
\node[system] (sistema) at (0,0) {\textbf{Sistema Pollen + Alexa}};

% Atores
\node[actor] (apicultor) at (-6,3) {Apicultor};
\node[actor] (alexa) at (6,3) {Amazon Alexa};

% Casos de uso principais
\node[usecase] (consultar_status) at (-3,2) {Consultar Status\\do Enxame};
\node[usecase] (registrar_alimentacao) at (0,2) {Registrar\\Alimentação};
\node[usecase] (verificar_colheita) at (3,2) {Verificar\\Colheita};
\node[usecase] (consultar_manejo) at (-3,0) {Consultar\\Manejo};
\node[usecase] (registrar_revisao) at (0,0) {Registrar\\Revisão};
\node[usecase] (consultar_dashboard) at (3,0) {Consultar\\Dashboard};
\node[usecase] (configurar_alexa) at (-3,-2) {Configurar\\Integração Alexa};
\node[usecase] (gerenciar_comandos) at (0,-2) {Gerenciar\\Comandos de Voz};
\node[usecase] (sincronizar_dados) at (3,-2) {Sincronizar\\Dados};

% Relacionamentos com Apicultor
\draw[->] (apicultor) -- (consultar_status);
\draw[->] (apicultor) -- (registrar_alimentacao);
\draw[->] (apicultor) -- (verificar_colheita);
\draw[->] (apicultor) -- (consultar_manejo);
\draw[->] (apicultor) -- (registrar_revisao);
\draw[->] (apicultor) -- (consultar_dashboard);
\draw[->] (apicultor) -- (configurar_alexa);
\draw[->] (apicultor) -- (gerenciar_comandos);

% Relacionamentos com Alexa
\draw[->] (alexa) -- (consultar_status);
\draw[->] (alexa) -- (registrar_alimentacao);
\draw[->] (alexa) -- (verificar_colheita);
\draw[->] (alexa) -- (consultar_manejo);
\draw[->] (alexa) -- (registrar_revisao);
\draw[->] (alexa) -- (consultar_dashboard);
\draw[->] (alexa) -- (sincronizar_dados);

% Relacionamentos de dependência entre casos de uso
\draw[->, dashed] (configurar_alexa) -- (gerenciar_comandos);
\draw[->, dashed] (gerenciar_comandos) -- (sincronizar_dados);
\draw[->, dashed] (sincronizar_dados) -- (consultar_status);
\draw[->, dashed] (sincronizar_dados) -- (registrar_alimentacao);
\draw[->, dashed] (sincronizar_dados) -- (verificar_colheita);

\end{tikzpicture}
\label{fig:caso-uso-alexa}
\end{figura}

A Figura \ref{fig:caso-uso-alexa} mostra que o sistema permite aos apicultores interagir com o Pollen através de comandos de voz, realizando consultas e registros de atividades apícolas de forma hands-free.

\subsecao{Estrutura de Dados do Sistema}

A estrutura de dados do sistema Pollen é composta por entidades principais que armazenam informações sobre usuários, enxames, atividades de manejo e produção apícola.

% Quadros descritivos da estrutura de dados - Substituindo DER

\begin{quadro}{Entidades e atributos do banco de dados relevantes para integração Alexa}{Elaborado pelo autor (2025)}
\label{quad:entidades-atributos}
\renewcommand{\arraystretch}{1.4}
\small
\resizebox{\textwidth}{!}{
\begin{tabular}{|l|p{7.5cm}|p{5.5cm}|}
\hline
\textbf{Entidade} & \textbf{Atributos} & \textbf{Descrição} \\ \hline

\textbf{USER} & 
\underline{id}, email, password, planType, status
& 
Armazena informações dos usuários do aplicativo Pollen que utilizarão a integração com Alexa \\ \hline

\textbf{ENXAME} & 
\underline{id}, userId, especie, estadoOrigem, localizacao, identificador, forcaEnxame, createddate
& 
Representa as colmeias gerenciadas pelo apicultor, principal entidade consultada via comandos de voz \\ \hline

\textbf{ALIMENTACAO} & 
\underline{id}, enxameId, tipo, quantidade, observacao, createddate
& 
Registra alimentações fornecidas aos enxames, pode ser consultada e registrada via Alexa \\ \hline

\textbf{COLHEITA} & 
\underline{id}, enxameId, quantidade, tipo, observacao, createddate
& 
Armazena informações sobre colheitas realizadas, principal métrica consultada via comandos de voz \\ \hline

\textbf{MANEJO} & 
\underline{id}, enxameId, tipo, descricao, createddate
& 
Registra atividades de manejo realizadas nas colmeias (divisão, troca de caixa, etc.) \\ \hline

\textbf{REVISAO} & 
\underline{id}, enxameId, tipo, observacao, createddate
& 
Armazena revisões periódicas realizadas nos enxames para verificação de saúde \\ \hline

\textbf{DASHBOARD} & 
\underline{id}, userId, totalEnxames, totalColheita, ultimaAtividade
& 
Consolida dados gerais do apiário, fornecendo resumo geral via comando de voz \\ \hline

\end{tabular}
}
\end{quadro}

\vspace{1em}

\begin{quadro}{Relacionamentos entre as entidades do sistema}{Elaborado pelo autor (2025)}
\label{quad:relacionamentos}
\renewcommand{\arraystretch}{1.5}
\resizebox{\textwidth}{!}{
\begin{tabular}{|l|l|l|p{7cm}|}
\hline
\textbf{Entidade Origem} & \textbf{Relacionamento} & \textbf{Entidade Destino} & \textbf{Descrição} \\ \hline

USER & possui (1:N) & ENXAME & 
Um usuário pode possuir múltiplos enxames. A Alexa consulta os enxames do usuário autenticado. \\ \hline

USER & visualiza (1:1) & DASHBOARD & 
Cada usuário possui um dashboard único com estatísticas gerais do seu apiário. \\ \hline

ENXAME & tem (1:N) & ALIMENTACAO & 
Um enxame pode ter múltiplos registros de alimentação ao longo do tempo. \\ \hline

ENXAME & produz (1:N) & COLHEITA & 
Um enxame pode ter múltiplas colheitas registradas. Principal dado consultado via Alexa. \\ \hline

ENXAME & recebe (1:N) & MANEJO & 
Um enxame pode ter múltiplos manejos realizados (divisão, troca de caixa, etc.). \\ \hline

ENXAME & inspecionado (1:N) & REVISAO & 
Um enxame pode ter múltiplas revisões periódicas para verificação de saúde. \\ \hline

\end{tabular}
}
\end{quadro}



O Quadro \ref{quad:entidades-atributos} apresenta as sete entidades principais do banco de dados Pollen acessadas pela Skill: User (usuários), Enxame (colmeias), Alimentacao (alimentação), Colheita (produção), Manejo (atividades), Revisao (inspeções) e Dashboard (estatísticas). Cada entidade possui atributos específicos para a gestão apícola.

O Quadro \ref{quad:relacionamentos} detalha como essas entidades se relacionam, demonstrando a estrutura de dados que permite à Alexa consultar informações de forma hierárquica. Por exemplo, quando o usuário pedir informações sobre seus enxames, a Skill acessa dados relacionados de alimentação, colheitas e revisões de cada enxame.

\secao{Interfaces do Sistema}
\label{sec:interfaces-sistema}

\subsecao{Wireframe - Autorização}

O wireframe de autorização mostra a tela onde o usuário concede permissões para que a Alexa acesse os dados do Pollen.

% Wireframe - Configuração da Integração Alexa
\begin{figura}{Wireframes - Configuração da integração Alexa no aplicativo Pollen}{O Autor}
\centering
\begin{tikzpicture}[scale=0.9]
% Definição de estilos
\tikzset{
  phone/.style={rectangle, draw, fill=gray!10, text width=3cm, text centered, minimum height=6cm},
  header/.style={rectangle, draw, fill=blue!20, text width=2.8cm, text centered, minimum height=0.6cm},
  button/.style={rectangle, draw, fill=green!30, text width=2.6cm, text centered, minimum height=0.5cm},
  text/.style={rectangle, draw, fill=white, text width=2.6cm, text centered, minimum height=0.4cm},
  switch/.style={rectangle, draw, fill=yellow!30, text width=2.6cm, text centered, minimum height=0.4cm}
}

% Tela 1: Configuração Inicial
\node[phone] (tela1) at (0,0) {};
\node[header] (header1) at (0,2.5) {\textbf{Configuração Alexa}};
\node[text] (text1) at (0,1.8) {Conecte sua conta Pollen\\com a Amazon Alexa};
\node[button] (btn1) at (0,1.2) {Conectar com Alexa};
\node[text] (text2) at (0,0.6) {Status: Não conectado};
\node[text] (text3) at (0,0.2) {Comandos disponíveis:\\• Consultar enxames\\• Registrar alimentação\\• Verificar colheita};
\node[button] (btn2) at (0,-0.4) {Próximo};
\node[button] (btn3) at (0,-0.9) {Cancelar};

% Tela 2: Autorização
\node[phone] (tela2) at (4,0) {};
\node[header] (header2) at (4,2.5) {\textbf{Autorização}};
\node[text] (text4) at (4,1.8) {A Alexa precisa de permissão\\para acessar seus dados};
\node[switch] (switch1) at (4,1.2) {$\checkmark$ Ler enxames};
\node[switch] (switch2) at (4,0.8) {$\checkmark$ Registrar alimentação};
\node[switch] (switch3) at (4,0.4) {$\checkmark$ Consultar colheita};
\node[switch] (switch4) at (4,0.0) {$\checkmark$ Acessar dashboard};
\node[button] (btn4) at (4,-0.6) {Autorizar};
\node[button] (btn5) at (4,-1.1) {Voltar};

% Tela 3: Configuração de Comandos
\node[phone] (tela3) at (8,0) {};
\node[header] (header3) at (8,2.5) {\textbf{Comandos de Voz}};
\node[text] (text5) at (8,1.8) {Configure como falar\\com a Alexa};
\node[text] (text6) at (8,1.2) {Exemplo: \textquotedblleft Alexa, pergunte\\ao Pollen sobre meus\\enxames\textquotedblright};
\node[text] (text7) at (8,0.6) {Comandos ativos:\\• Status do enxame\\• Registrar alimentação\\• Verificar colheita};
\node[button] (btn6) at (8,0.0) {Testar Comando};
\node[button] (btn7) at (8,-0.5) {Finalizar};
\node[button] (btn8) at (8,-1.0) {Voltar};

% Setas de navegação
\draw[->] (tela1) -- node[above] {Próximo} (tela2);
\draw[->] (tela2) -- node[above] {Autorizar} (tela3);

\end{tikzpicture}
\label{fig:wireframe-configuracao-alexa}
\end{figura}

A Figura \ref{fig:wireframe-autorizacao-alexa} apresenta a tela onde o usuário visualiza e autoriza as permissões necessárias. As Figuras \ref{fig:wireframe-comandos-disponiveis} e \ref{fig:wireframe-testar-comandos} mostram a lista de comandos disponíveis e a interface para testar comandos antes de usar com a Alexa.

\subsecao{Wireframes - Gerenciamento de Comandos}

% Wireframe - Todos os Comandos Alexa
\begin{figura}{Wireframe - Tela com todos os comandos disponíveis da integração Alexa}{Elaborado pelo autor (2025)}
  \centering
  \includegraphics[width=0.4\textwidth]{resources/floats/ilustracoes/todos_comandos_alexa_wireframe.png}
  \label{fig:wireframe-todos-comandos}
\end{figura}

% Wireframe - Desconectar Alexa
\begin{figura}{Wireframe - Tela para desconectar a integração com Alexa}{Elaborado pelo autor (2025)}
  \centering
  \includegraphics[width=0.4\textwidth]{resources/floats/ilustracoes/desconectar_alexa_wireframe.png}
  \label{fig:wireframe-desconectar-alexa}
\end{figura}


A Figura \ref{fig:wireframe-todos-comandos} exibe a tela com todos os comandos de voz disponíveis, permitindo que o usuário conheça as funcionalidades acessíveis por voz. A Figura \ref{fig:wireframe-desconectar-alexa} apresenta a interface para desconectar a integração, caso o usuário deseje revogar as permissões.

\secao{Fluxo de Processamento do Sistema}
\label{sec:fluxo-processamento}

\subsecao{Processo de Execução de Comandos de Voz}

% Quadro - Processo de Comandos de Voz Alexa
\begin{quadro}{Processo de execução de comandos de voz na integração Alexa}{Elaborado pelo autor (2025)}
\label{quad:fluxo-comandos-voz}
\renewcommand{\arraystretch}{1.5}
\small
\begin{tabular}{|c|p{4cm}|p{8cm}|}
\hline
\textbf{Etapa} & \textbf{Processo} & \textbf{Descrição} \\ \hline

1 & Captura de Comando & O usuário fala um comando para o dispositivo Alexa. Exemplo: \textquotedblleft Alexa, pergunte ao Pollen quantos enxames eu tenho\textquotedblright \\ \hline

2 & Reconhecimento de Voz & A Alexa reconhece e processa o comando utilizando processamento de linguagem natural (NLP) \\ \hline

3 & Validação do Comando & O sistema verifica se o comando solicitado é válido e está disponível na Skill Pollen. Se inválido, a Alexa fornece ajuda sobre comandos disponíveis \\ \hline

4 & Verificação de Autenticação & O sistema verifica se o usuário está autenticado e vinculado a uma conta Pollen. Se não autenticado, solicita vinculação da conta \\ \hline

5 & Chamada à API & Com o usuário autenticado, o sistema realiza chamada à API do Pollen para buscar os dados solicitados (enxames, colheitas, dashboard, etc.) \\ \hline

6 & Processamento de Dados & Os dados retornados pela API são processados e formatados em linguagem SSML (Speech Synthesis Markup Language) para resposta em voz \\ \hline

7 & Resposta ao Usuário & A Alexa fornece a resposta em voz ao usuário com as informações solicitadas de forma clara e objetiva \\ \hline

\end{tabular}
\end{quadro}

O Quadro \ref{quad:fluxo-comandos-voz} apresenta as sete etapas do processo de execução de comandos de voz. O processo inicia com a captura do comando pelo dispositivo Alexa, passa pelo reconhecimento de linguagem natural (NLP), validação do comando e autenticação do usuário. Após a autenticação, o sistema chama a API do Pollen para buscar os dados solicitados, processa esses dados formatando-os em SSML (Speech Synthesis Markup Language) e retorna a resposta em voz. Este fluxo garante que apenas usuários autenticados acessem informações sensíveis e que comandos inválidos recebam mensagens de ajuda adequadas.

\secao{Considerações Finais do Capítulo}

A solução planejada representa uma integração inovadora entre assistentes virtuais e aplicações de gestão apícola, oferecendo aos apicultores uma forma eficiente e hands-free de acessar informações durante o trabalho no apiário.

Os artefatos de planejamento demonstram a viabilidade técnica da integração. O Diagrama de Caso de Uso ilustra as funcionalidades principais e interações entre os atores. Os Quadros de estrutura de dados definem as entidades e relacionamentos usados pela integração. Os Wireframes apresentam as interfaces de configuração no aplicativo Pollen. O Quadro de Processo de Comandos de Voz detalha as sete etapas de execução.

O planejamento atende aos requisitos funcionais e não funcionais da Seção \ref{sec:requisitos-sistema}, proporcionando base sólida para a futura implementação no TCC 02. A integração contribuirá para a eficiência da gestão apícola, permitindo acesso à informação através de comandos de voz naturais em português brasileiro.