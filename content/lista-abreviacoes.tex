% Lista de Abreviações e Siglas
\makeatletter
\ut@section[cap:lista-abreviacoes]{ut@capitulo}{\bfseries\MakeUppercase}{Lista de Abreviações e Siglas}
\makeatother

% Configuração de espaçamento conforme orientações do professor
\renewcommand{\baselinestretch}{1.5}\selectfont
\setlength{\parskip}{0pt}

\begin{flushleft}
\addcontentsline{toc}{section}{Lista de Abreviações e Siglas}

\textbf{ABNT} - Associação Brasileira de Normas Técnicas

\textbf{API} - Application Programming Interface

\textbf{AWS} - Amazon Web Services

\textbf{CNN} - Convolutional Neural Network


\textbf{HTTP} - Hypertext Transfer Protocol

\textbf{IA} - Inteligência Artificial

\textbf{JSON} - JavaScript Object Notation

\textbf{JWT} - JSON Web Token

\textbf{NLP} - Natural Language Processing

\textbf{Node.js} - Node.js Runtime Environment

\textbf{ONU} - Organização das Nações Unidas

\textbf{PDF} - Portable Document Format

\textbf{RAM} - Random Access Memory

\textbf{REST} - Representational State Transfer

\textbf{SDK} - Software Development Kit

\textbf{SSML} - Speech Synthesis Markup Language

\textbf{TCC} - Trabalho de Conclusão de Curso

\textbf{UML} - Unified Modeling Language

\textbf{URL} - Uniform Resource Locator


\end{flushleft}

% Restaurar configurações padrão
\renewcommand{\baselinestretch}{1.0}\selectfont
\setlength{\parskip}{\baselineskip}
